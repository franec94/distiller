



\begin{frame}
\frametitle{Experiments Design (2)}
We decided to organize and design our experiments following a strict and precise series of subsequent steps that allow us to break down the different targets and goals
we identified for accomplishing the case study we want to pursue. In particular we proceed in the following manner:

\begin{itemize}
\item After haivng produced at least a bounch of datasets planty of data points representing examples of control group we collected data for different compressing algorithm.
In particular we focused on:
\begin{itemize}
\item \textbf{\textit{Automated Gradual Pruning}} by \textit{Michael Zhu, Suyog Gupta et al., 2017)} - as an instance of a possible pruning like compressing method;
\item \textbf{\textit{Linear Range Quantization}} by \textit{ Benoit et al., 2018 } - a \textit{Quantization Aware Training Technique} which support \textbf{{\textit{Symmetric, Asimmetric both Signed and Unsigned features}}
along with the possibility of declaring the \textbf\textit{level of quantization}} to which set each layer of a distinct deep model.
\end{itemize}

\end{itemize}
\end{frame}