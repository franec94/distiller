



\begin{frame}
\frametitle{Linear Range Quant}
\textbf{Automated Gradual Pruner (AGP)}, discussed in \textbf{To prune, or not to prune: exploring the efficacy of pruning for model compression}, the authors Michael Zhu and Suyog Gupta propose an intuitive, fresh new pruning approach briefly described as:

\begin{itemize}
\item \textbf{new automated gradual pruning algorithm} in which the sparsity is increased from an initial sparsity value $s_{i}$ (usually 0) to a final sparsity value $s_{f}$ over a span of $n$ pruning steps;
\item The intuition behind this sparsity function in equation below is to \textbf{prune the network rapidly in the initial phase} when:
\begin{itemize}
\item the redundant connections are abundant, and;
\item gradually reduce the number of weights being pruned each time as there are fewer and fewer weights remaining in the network.
\end{itemize}
\end{itemize}

% \begin{equation}
% s_{t} = s_{f} + (s_{i} - s_{f}(1 - \frac{t-t_{0}}{n \Delta t})^{3}), t \in \{t_{0}, t_{0} + \Delta t, \dots, t_{0} + \Delta n\}
% \end{equation}


\end{frame}

\begin{frame}
\frametitle{Linear Range Quant (2)}
Other interesting properties related to \textbf{Automated Gradual Pruner (AGP)}, and discussedwithin Michael Zhu and Suyog Gupta's paper, for better appreciating and understanding the usefulness of such a technique are:

\begin{itemize}
\item It Requires a lower number of trials, and attempts for identifying meaningful set of hyper-params for leading the pruning approach, compared to other techniques such Magnitude Level Pruning and other similars;
\item It does not made particular assumptions on weight values density distribution;
\item It is agnostic with respect to the particular Deep Neural Network Architecture chosen;
\end{itemize}



\end{frame}
