



\begin{frame}
    \frametitle{Selected Deep Nets Compression Techniques}
        % \centering \Huge
        \begin{center}
            {\fontsize{40}{50}\selectfont \emph{Conclusions}}
        \end{center}
        \begin{center}
            \emph{Last Remarks learnt from compressing Siren Based Nets}
        \end{center}
\end{frame}


\begin{frame}
    \frametitle{Conclusions and Remarks}
    The major remarks we have found while appling Deep Networks Compression Techiniques such as pruning-based approaches and quantizing-based approaches
    are the following. In particular, speaking about adopted Pruning Algorithm, which is Automated Gradual Pruning (AGP) we notice that:
    \begin{itemize}
        \item Models seem to need a reasonable extensive number of epochs to improve over brain damage, i.e. weights prunining
        \item It result a better choice to let models to be pruned when an adequate large number of steps last between two consecutive wheight pruning
        \item item We were able to identify Hyper-params combinations for AGP pruning method that
        \begin{itemize}
            \item outperform baseline architectures, i.e. plain Siren-based Nets.
            \item satisty constraints of being at least lower in size w.r.t Learned
            Image as well as still performing better than some comparing in size plain Siren-nets.
        \end{itemize}
        \item However, when looking at performance that should be compared to Jpeg compression results we still notice that the existing gap
        between values related to Bpp score are too large to justify just such a technique to effectively learn an implictly representation of the Target Image.
    \end{itemize}
\end{frame}


\begin{frame}
    \frametitle{Conclusions and Remarks (2)}
    The major remarks we have found while appling Deep Networks Compression Techiniques such as pruning-based approaches and quantizing-based approaches
    are the following. Where, talking about adopted Quant Procedure, which is Linear Range quant-aware training we claim that:
    \begin{itemize}
        \item We have quantized starting from already trained and pruned models, obtaining in best cases that:
        \begin{itemize}
            \item I wide reduction in Bpp score corresponded a drammaticaly reduction in Pnsr values
            \item Still we have retrieved quantized model instances that show an acceptable Psnr score value w.r.t both Jpeg compressed counterpart,
            and even an indisputable results wr.t plain Siren-based models.
        \end{itemize}
        \item However, Learning Rate Hyper-param was crucial for identifying promising Linear Range Quant Configurations, and lower Learning Rate values
        seems to outperform larger ones.
        \item Differently from AGP pruning technique, we observed that for training by means of quant awere procedures as LRQ:
        \begin{itemize}
            \item it was better to increase frequency with which updating learnable parameters to let quantization to take place
            \item it was better to halt training as soon as possible in order to do not let weigths varying such that their density distribution will not
            be any more adequante w.r.t learned quantization parameters.
        \end{itemize}        
    \end{itemize}
\end{frame}
